\documentclass{article}
\usepackage{float}
\usepackage[pdftex]{graphicx}
\usepackage{listings}
\usepackage{xcolor}
\usepackage[cm]{fullpage}
\lstset{basicstyle=\ttfamily,
  showstringspaces=false,
  commentstyle=\color{red},
  keywordstyle=\color{blue}
}
\begin{document}

\author{Jon Robison}
\title{CS595 Assignment 10}
\maketitle

Q1.\\
Choose a blog or a newsfeed (or something similar with an Atom or RSS feed).  It should be on a topic or\\
topics of which you are qualified to provide classification training data. Find something with at least \\
100 entries Create between four and eight different categories for the entries in the feed:..download and process.\\*

The program getTitles.py retrieves the paginated results and outputs to a pipe separated text\\
file. This file is then manually processed to include necessary training information. This is \\
accomplished by putting the feature in the 3rd location and the actual category in the 4th \\
location (after the third pipe) when appropriate.\\*

See program given in Appendix A.\\*

Q2.\\
Manually classify the first 50 entries, and then classify (using the fisher classifier) the remaining\\
50 entries. Create a table with the title, the string used for classification, cprob(), predicted category,\\
and actual category.\\*

The driver program invokes docclass, reading the file given from A. This trains the classifier with the\\
first half of whatever is given in the file, and guesses the last half. Different results are printed to a\\
latex file and the console for the results table and debugging information/status, respectively.\\*

\begin{tabular}{ l l l l l }
\tiny Title & \tiny Feature & \tiny Predicted & \tiny Category & \tiny CP\\
\tiny ALBUM: FALL CITY FALL - VICTUS& \tiny FALL CITY& \tiny album& \tiny album& \tiny 0.0\\ 
\tiny ALBUM: HANDBOOK - TITANOMACHY& \tiny HANDBOOK& \tiny album& \tiny album& \tiny 0.0\\ 
\tiny ALBUM : WIHT - THE HARROWING OF THE NORTH& \tiny WIHT& \tiny album& \tiny album& \tiny 0.0\\ 
\tiny SINGLE: IVORY SEAS - STILL BROODING / MOTHERS TONGUE& \tiny IVORY SEAS& \tiny single& \tiny single& \tiny 0.0\\ 
\tiny EXCLUSIVE: ALCOPOP! Cat in a Hat 2013 Compilation \#30& \tiny Cat in& \tiny event& \tiny event& \tiny 0.0\\ 
\tiny SINGLE: MARIKA HACKMAN - BATH IS BLACK& \tiny MARIKA HACKMAN& \tiny single& \tiny single& \tiny 0.0\\ 
\tiny FUTURE SOUNDS OF 2013& \tiny FUTURE SOUNDS& \tiny event& \tiny event& \tiny 0.0\\ 
\tiny THE BLOG SOUND OF 2013 - THE TOP 5& \tiny BLOG SOUND& \tiny event& \tiny event& \tiny 0.0\\ 
\tiny MUSIC LIBERATION TOP 10 ALBUMS OF 2012& \tiny TOP 10& \tiny event& \tiny event& \tiny 0.0\\ 
\tiny Merry Christmas 2012 : First Aid Kit - Blue Christmas& \tiny Merry Christmas& \tiny event& \tiny event& \tiny 0.0\\ 
\tiny MUSIC LIBERATION - TOP 10 EPS OF 2012& \tiny EP'S OF& \tiny event& \tiny event& \tiny 0.0\\ 
\tiny MUSIC LIBERATIONS SINGLE OF THE YEAR 2012& \tiny OF THE YEAR& \tiny event& \tiny event& \tiny 0.0\\ 
\tiny THE BLOG SOUND OF 2013& \tiny THE BLOG& \tiny event& \tiny event& \tiny 0.0\\ 
\tiny NEW MUSIC: MARIKA HACKMAN& \tiny MARIKA HACKMAN& \tiny new& \tiny new& \tiny 0.0\\ 
\tiny EP: NINETAILS - SLEPT AND DID NOT SLEEP& \tiny NINETAILS& \tiny album& \tiny album& \tiny 0.0\\ 
\tiny EP: TO KILL A KING - WORD OF MOUTH& \tiny KILL A KING& \tiny album& \tiny album& \tiny 0.0\\ 
\tiny EP: OLILOQUY - THE RED EP& \tiny THE RED& \tiny album& \tiny album& \tiny 0.0\\ 
\tiny ALBUM: CITIZENS - CTZNS& \tiny CITIZENS& \tiny album& \tiny album& \tiny 0.0\\ 
\tiny ALBUM: SUP PEEPS - START COLLECTING& \tiny SUP PEEPS& \tiny album& \tiny album& \tiny 0.0\\ 
\tiny ALBUM: TALL SHIPS - EVERYTHING TOUCHING& \tiny EVERYTHING TOUCHING& \tiny album& \tiny album& \tiny 0.0\\ 
\tiny ALBUM: NEGATIVE PEGASUS - LOOMING& \tiny NEGATIVE PEGASUS& \tiny album& \tiny album& \tiny 0.0\\ 
\tiny SINGLE: LAURA WELSH - CALL TO ARMS / HOLLOW DRUM& \tiny LAURA WELSH& \tiny single& \tiny single& \tiny 0.0\\ 
\tiny EP: CARA MITCHELL - HAVE YOU EVER WONDERED& \tiny CARA MITCHELL& \tiny album& \tiny album& \tiny 0.0\\ 
\tiny SINGLE REVIEW: DEAD WOLF CLUB - RADAR& \tiny DEAD WOLF CLUB& \tiny album& \tiny single& \tiny 0.0\\ 
\tiny LIVE REVIEW : Lucy Rose at the Jazz Cafe (Camden, London, 4th February 2012)& \tiny Lucy Rose at& \tiny album& \tiny event& \tiny 0.0\\ 
\tiny Music Liberation and ooShirts.com& \tiny and ooShirts.com& \tiny interview& \tiny event& \tiny 0.0\\ 
\tiny ALBUM REVIEW: CLOUD NOTHINGS - ATTACK ON MEMORY& \tiny CLOUD NOTHINGS& \tiny album& \tiny album& \tiny 0.0\\ 
\tiny New Music : The Cast Of Cheers& \tiny Cast Of Cheers& \tiny interview& \tiny new& \tiny 0.0\\ 
\tiny Future Sounds of 2012& \tiny Future& \tiny event& \tiny event& \tiny 0.0\\ 
\tiny Music Liberations Writers takeover week : Moker& \tiny Writers takeover& \tiny album& \tiny event& \tiny 0.0\\ 
\tiny Music Liberations Writers takeover week : Clive Rozario& \tiny Clive Rozario& \tiny album& \tiny event& \tiny 0.0\\ 
\tiny Music Liberations Writers takeover week : Chris Meredith& \tiny Chris Meredith& \tiny album& \tiny event& \tiny 0.0\\ 
\tiny Music Liberations Writers takeover week : Tom Nash& \tiny Tom Nash& \tiny album& \tiny event& \tiny 0.0\\ 
\tiny The Blog Sound of 2012 : The Top 5& \tiny Top 5& \tiny album& \tiny event& \tiny 0.0\\ 
\tiny Music Liberations Top 10 Albums of 2011& \tiny Albums of& \tiny album& \tiny event& \tiny 0.0\\ 
\tiny Music Liberations Top 5 EPs of 2011& \tiny EPs of 2011& \tiny album& \tiny event& \tiny 0.0\\ 
\tiny Music Liberations Top 5 Tracks of 2011& \tiny Tracks of 2011& \tiny album& \tiny event& \tiny 0.0\\ 
\tiny Merry Christmas 2011& \tiny Christmas 2011& \tiny album& \tiny event& \tiny 0.0\\ 
\tiny EP Review : Likes Lions - Future Colour& \tiny Likes Lions& \tiny album& \tiny album& \tiny 0.0\\ 
\tiny EP Review : FaltyDL - Atlantis& \tiny FaltyDL& \tiny album& \tiny album& \tiny 0.0\\ 
\tiny EP Review : Sea Oleena - Sleeplessness& \tiny Sea Oleena& \tiny album& \tiny album& \tiny 0.0\\ 
\tiny EP Review : Alphabet Backwards - British Explorer& \tiny British Explorer& \tiny album& \tiny album& \tiny 0.0\\ 
\tiny EP Review : Daughter - The Wild Youth& \tiny Wild Youth& \tiny album& \tiny album& \tiny 0.0\\ 
\tiny New Video : Gabrielle Aplin - Home& \tiny Home& \tiny album& \tiny new& \tiny 0.0\\ 
\tiny The Blog Sound of 2012& \tiny Blog Sound of 2012& \tiny event& \tiny event& \tiny 0.0\\ 
\tiny Cutting Noise Podcast 002 (November 2011)& \tiny Cutting Noise& \tiny album& \tiny event& \tiny 0.0\\ 
\tiny Album Review : Hectic Zeniths - Hectic Zeniths& \tiny Hectic Zeniths& \tiny album& \tiny album& \tiny 0.0\\ 
\tiny New Music : Rosa Valle / Therapist / Smother Party& \tiny Therapist& \tiny album& \tiny new& \tiny 0.0\\ 
\tiny New Video : The Producers - Episode 2: The Bullitts& \tiny The Producers& \tiny album& \tiny new& \tiny 0.0\\ 
\tiny Album Review : Evidence - Cats \& Dogs& \tiny Evidence& \tiny album& \tiny album& \tiny 0.0\\ 
\end{tabular}

\\*

CProb is notably low, indicating low confidence in the guess. My believe is that since the blog is a music\\
blog, and the titles are typically band or album names, the classifier can't distinguish titles. I initially\\
went with this blog since it already broke down the classifications for me, but realized upon identifying the\\
features it wouldn't be great since the classications were the only unique data. That would be my next step.\\*

See program given in Appendix B and C.\\*

Q3.\\
Assess the performance of your classifier in each of your categories by computing precision and recall.  Note\\
that the definitions are slightly different in the context of classification.\\*

\begin{tabular}{ l l l }
	&Precision	&Recall\\
album	&19/34		&19/19\\
new	&1/1		&1/5\\
event	&10/10		&10/22\\
single	&3/3		&3/4\\
interview&0/2		&0/0\\
\end{tabular}

\appendix
\newpage
Appendix A
\lstinputlisting[language=python]{q1/getTitles.py}

\newpage
Appendix B
\lstinputlisting[language=python]{q2/driver.py}

Appendix C
\lstinputlisting[language=python]{q2/docclass.py}
\end{document}
